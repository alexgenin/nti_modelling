\documentclass[a4paper,11pt]{article}
\usepackage[utf8]{inputenc}
\usepackage[T1]{fontenc}
\usepackage{textcomp}
\usepackage{amsmath}
% \usepackage{datetime} % for \date command
\usepackage{rotating}

\input{/home/alex/docs/resources/latex_include.tex}
\usepackage[english]{babel}

\usepackage{geometry}
\geometry{
  left   = 24mm,
  right  = 24mm,
  top    = 24mm,
  bottom = 24mm,
}


% bib
\usepackage[style=numeric-comp,
            doi=false,
            url=false]{biblatex}
\bibliography{/home/alex/docs/resources/latex/library.bib}

%opening
\title{Non-trophic interactions: consequences on secondary extinctions}
\author{Alexandre Génin}

\begin{document}
  
  \maketitle 
  
  \section{The trophic model}
      
      \SCfig{trophic_topology.png}{
        The fixed trophic topology: 4 basal producers, 2 grazers and 2 
        consumers. Species consume everyone on the lower trophic level.
        x-axis has no meaning.}{.6}
      
      The node dynamics model is inspired from \autocite{Brose2006}. It is 
given by its main equation that describes the dynamics of one species' biomass: 
    
    \begin{equation}
      \label{eq:popdyn}
      \frac{dB_i}{dt} = \underbrace{rB_i(1 - \frac{B_i}{K_i})}_{logistic\ growth}
                        \underbrace{+ \sum_{k=1}^{N} B_i w_{ik} F_{ik}}_{i\ eats}
                        \underbrace{- \sum_{j=1}^{N} B_j w_{ji} F_{ji} / e_{ji}}_{i\ gets\ eaten}
                        \underbrace{- x_i B_i}_{mortality}
    \end{equation}
    
    where $F_{ij}$ describes a generic functional response in which species $i$ 
feeds on $j$ as followed: 
    
    \begin{equation}
      F_{ij} = \frac{a_{ij} w_{ij} B_{j}^{q+1} }{ 
                       1 + h_{i} w_{ij} \sum_{k} a_{ik} B_{k}^{q+1} }
    \end{equation}
    
    Parameters are described in Table \ref{tab:parameters}. In the functional 
response, q is a coefficient taken randomly in the $[0,1]$ range (a value of 0 
yields a type-II functional and 1 a type-III functional response).
    
    Many of these parameters can be chosen using metabolic scaling rules 
resulting in a plausible food web with few free parameters. Details on the 
default parameter values can be found in Table \ref{tab:parameters}. The trophic 
topology (values for which $a_{ij}>0$) is held constant.
    
    \begin{sidewaystable}[s]
      \begin{tabular}{r | l | l}
        Param.    & Comment                                               & Value  \\
        \hline 
        $r_i$     & Reproductive rate of species $i$                      & 1 for producers, 0 otherwise \\
        $K_i$     & The carrying capacity of species $i$                  & 1 for all producers \\
        $w_{ij}$  & The consumption rate of species $j$ by species $i$    & equal between all preys (e.g. 0.25 for grazers (4 preys)) \\
        $F_{ij}$  & The functional response of species $i$ on $j$         & \\
        $q$       & The ``hill'' coefficient in the functional response   & random between 0 and 1 \\
        $e_{ji}$  & The conversion efficiency of species $i$ into $j$     & 0.85 for all $i$ and $j$ \\
        $b_{i}$   & \textbf{Body mass of species $i$}                     & producers: 1, grazers: 3 predators: 6 \\
        $x_i$     & The metabolic (mortality) rate                        & scaling rule with body size $x_i = 0.223 b_i^{-.25}$ \\
        $a_{ij}$  & The attack rate of species $i$ on $j$                 & for species $i$ attacking $j$: $27.24 * x_{i} * exp(0.01 * b_{i}/b_{j})$ \\
        $h_i$     & The handling time of species $i$                      & 1/(8*$x_i$) 
      \end{tabular}
      \caption{Default parameters and values used in the model. Some of them (e.g. $x_i$ use metabolic scaling relationships). }
      \label{tab:parameters}
    \end{sidewaystable}
    
    \subsection{Results}
      \subsubsection{Example output}
      
        The simulation process is as follow: 
        \begin{itemize}
          \item Species initial biomasses are chosen randomly in the range 
                $]0;K_i]$, q is chosen randomly between $0$ and $1$.
          \item The simulation is run until $t=3000$ is reached, when one or 
                more species are removed. The run lasts until $t=5000$ is 
                reached. 
        \end{itemize}
        
        \afig{trophic_sample_output.png}{100 replicates. sp1-4 are producers, 
sp5-6 are intermediate consumers and sp7-8 are top predators. In this 
simulation, species 5 (grazer) is removed at time $t=3000$. Mind the non-linear 
vertical scale}{1}{!h}
  
  \section{Generalization of NTIs}
    
    The introduction of a \emph{generic} non-trophic interaction can be done 
by adding a term to the equation of a node (Eq. \ref{eq:popdyn}).
    
  \begin{equation}
    \label{eq:popdynt}
    \frac{dB_i}{dt} = rB_i(1 - \frac{B_i}{K_i})
                      + \sum_{k=1}^{N} B_i w_{ik} F_{ik}
                      - \sum_{j=1}^{N} B_j w_{ji} F_{ji} / e_{ji}
                      - x_i B_i 
                      + \underbrace{\sum_{j=1}^{N} N^t_{ij} B_j}_{non-trophic\ interactions}
  \end{equation}
    
    Note that the non-trophic interaction is constrained to be always below the 
``mortality'' rate ($- x_i B_i$) so that there is no creation of biomass from 
non-trophic interactions.
    
  \newpage
%   
%   \section*{TODO}
%     \begin{itemize}
%       \item Think about how to set the body mass ratios between preys and 
%             predators.
%       \item Check and think about metabolic relationships (ectoterm/endotherm, 
%             that paper about metabolism of plants)
%       \item Think about which ones should be free and which ones should be 
%             fixed.
%       \item Think about the efficiency: rate of prey -> biomass or 
%     \end{itemize}
%   
  
  \newpage
  \begin{footnotesize}
    \printbibliography
  \end{footnotesize}
  
\end{document}

