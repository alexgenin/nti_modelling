\documentclass[a4paper,11pt]{article}
\usepackage[utf8]{inputenc}
\usepackage[T1]{fontenc}
\usepackage{textcomp}
\usepackage{amsmath}
% \usepackage{datetime} % for \date command
\usepackage{rotating}

\input{/home/alex/docs/resources/latex_include.tex}
\usepackage[french]{babel}

\usepackage{geometry}
\geometry{
  left   = 24mm,
  right  = 24mm,
  top    = 24mm,
  bottom = 24mm,
}


% bib
\usepackage[style=numeric-comp,
            doi=false,
            url=false]{biblatex}
\bibliography{/home/alex/docs/resources/latex/library.bib}

%opening
\title{Non-trophic interactions: consequences on secondary extinctions}
\author{Alexandre Génin}

\begin{document}
  
  \maketitle 
  
  \section{The trophic model}
      
      \SCfig{trophic_topology.png}{
        The fixed trophic topology: 4 basal producers, 2 grazers and 2 
        consumers. Species consume everyone on the lower trophic level.
        x-axis has no meaning.}{.6}
      
      The dynamic model is inspired from Brose et al's (2006). It is given by 
its main equation that describes the dynamics of one species' biomass: 
    
    \begin{equation}
      \label{eq:popdyn}
      \frac{dB_i}{dt} =   r_iB_i(1 - \frac{B_i}{K_i}) 
                        + \sum_{i=1}^{i=N} B_i w_{ij} F_{ij} 
                        - \sum_{j=1}^{i=N} B_j w_{ji} F_{ji} / e_{ji} 
                        - x_i B_i 
    \end{equation}
    
    where $F_{ij}$ describes a generic functional response in which species $i$ 
feeds on $j$: 
    
    \begin{equation}
      F_{ij} = \frac{a_{ij} w_{ij} B_{j}^{q+1} }{ 
        1 + h_{i} w_{ij} \sum_{k} a_{ik} B_{k}^{q+1} }
    \end{equation}
    
    Parameters are described in Table \ref{tab:parameters}. In the functional 
response, q is a coefficient taken randomly between $0$ (type-II functional 
response) and $1$ (type-III functional response). 
    
    Many of these parameters can be chosen using metabolic scaling rules 
yielding a model with few free parameters \footnote{Note that we need to think 
about how closely we want to follow metabolic scaling rules [\today]}       .
    
    Details on the default parameter values can be found in Table 
\ref{tab:parameters}. The trophic topology (values for which $a_{ij}>0$) is held 
constant.
    
    \begin{sidewaystable}[s]
      \begin{tabular}{r | l | l}
        Param.    & Comment                                               & Value  \\
        \hline 
        $r_i$     & Reproductive rate of species $i$                      & 1 for producers, 0 otherwise \\
        $K_i$     & The carrying capacity of species $i$                  & 1 for all producers \\
        $w_{ij}$  & The consumption rate of species $j$ by species $i$    & equal between all preys (e.g. 0.25 for grazers (4 preys)) \\
        $F_{ij}$  & The functional response of species $i$ on $j$         & \\
        $q$       & The ``hill'' coefficient in the functional response   & random between 0 and 1 \\
        $e_{ji}$  & The conversion efficiency of species $i$ into $j$     & 0.85 for all $i$ and $j$ \\
        $b_{i}$   & \textbf{Body mass of species $i$}                     & producers: 1, grazers: 3 predators: 6 \\
        $x_i$     & The metabolic (mortality) rate                        & scaling rule with body size $x_i = 0.223 b_i^{-.25}$ \\
        $a_{ij}$  & The attack rate of species $i$ on $j$                 & scaling rule with \\
        $h_i$     & The handling time of species $i$                      & 1/(8*$x_i$) 
      \end{tabular}
      \caption{Default parameters and values used in the model. Some of them 
(e.g. $x_i$ use metabolic scaling relationships). }
      \label{tab:parameters}
    \end{sidewaystable}
    
    \subsection{Results}
      \subsubsection{Example output}
      
        The simulation process is as follow: 
        \begin{itemize}
          \item Species initial biomasses are chosen randomly in the range 
                $]0;K_i]$, q is chosen randomly between $0$ and $1$.
          \item The simulation is run until $t=3000$ is reached, when a species 
                is removed. The run lasts until $t=5000$ is reached. 
        \end{itemize}
        
        \afig{trophic_sample_output.png}{100 replicates. sp1-4 are producers, 
sp5-6 are intermediate consumers and sp7-8 are top predators. In this 
simulation, species 5 (grazer) is removed at time $t=3000$. Mind the non-linear 
vertical scale}{1}{!h}
  
  \section{An example of non-trophic interaction}
    All the parameters of the model here are fixed beforehand and do not depend 
on the biomasses of species. Non-trophic interactions can be implemented by 
introducing that dependence on species abundances.
    
    For example, let's consider species $i$, a producer: its logistic growth is 
controlled by its fixed carrying capacity $K$. However, let's consider that some 
species from upper trophic levels create new space for algaes to grow on, thus 
increase its value (e.g. mussels/propagules). 
    
    Instead of a fixed value $K$ in Eq. \ref{eq:popdyn}, we replace it by 
$K_{i}$ that depends on interactions with other species: 
    
    \begin{equation}
      K_{i} = \sum_{j=1|\delta K_{ij}\neq0}^{N} \frac{ K_0 B_0 + (K_0 + \delta K_{ij}) B_j }{ B_0 + B_j }
    \end{equation}
    
    $\delta K_{ij}$ represents the bonus (if positive), or penalty (if negative) 
on parameter $K$ that species $i$ receives from species $j$. 
    
    \afig{ntrophic_sample_output.png}{Example output for 100 simulations, where 
the carrying capacity $K$ of producers (sp1-4) depend on the abundance of 
species 5 i.e. $\delta K_{5(1,2,3,4)}$ is set to $0.3$. In this simulation, species 
5 (grazer) is removed at time $t=3000$. Notice the higher drop in predators 
abundances (sp7-8) compared to \ref{fig:trophic_sample_output.png}, and the 
absence of secondary extinctions.}{1}{!h}
  
  \section{``Generalization`` of NTIs}
    This approach can be generalized to any parameter $p$: however, not all 
parameters can be changed in a way that makes sense. For example, the 
grazers and predators have no $K$ thus the above example is not easily 
generalizable.
    
    A possibility would be to introduce a generic coefficient of bonus/penalty 
in trophic interactions, on $w_{ij}$ for instance. 
  
  \newpage
  \section*{TODO}
    \begin{itemize}
      \item Why are top predators not identical ? due to different initial ab
      \item Think about how to set body masses
      \item Check and think about metabolic relationships
      \item So far, parameters are chosen a bit arbitrarily: think about which 
            ones should be free and which ones should be fixed.
    \end{itemize}
  
  
  \newpage
  \begin{footnotesize}
    \printbibliography
  \end{footnotesize}
  
\end{document}
